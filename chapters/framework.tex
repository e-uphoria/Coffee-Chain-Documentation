\chapter{Foundational Framework and Reference Model}

The foundation of the Coffee Chain ERP lies in its structured 
design, which defines how outlets, sales, CRM, and menu items 
interact. This base serves as the frame of reference for users, 
developers, and managers.

\section*{Core Components}
\begin{itemize}
    \item \textbf{Outlets:} Defined by outlet name, location, manager, 
    and regional manager. Outlets serve as the primary business units 
    for tracking performance.
    \item \textbf{Menu:} Menu items are created in the Coffee Menu 
    module and automatically integrated into the Sales product list. 
    This ensures a centralized definition of products.
    \item \textbf{Sales:} Handles quotations, orders, and invoicing. 
    Reporting provides key performance metrics for outlet evaluation.
    \item \textbf{CRM:} Extended to track customer leads associated 
    with specific outlets, improving targeted sales efforts.
\end{itemize}

\section*{Reference Framework}
The ERP aligns with established business management practices, 
including:
\begin{itemize}
    \item \textbf{PDCA Cycle (Plan–Do–Check–Act):} Outlets plan and 
    execute sales, monitor outcomes through reporting, and implement 
    improvements.
    \item \textbf{QMS Principles:} Clear responsibilities (manager and 
    regional manager) and standardized workflows ensure consistent 
    quality across outlets.
    \item \textbf{SIPOC Model:}
    \begin{itemize}
        \item \textbf{Suppliers:} Coffee menu, suppliers, and managers
        \item \textbf{Inputs:} Menu items, outlet data, customer leads
        \item \textbf{Process:} Sales and CRM operations
        \item \textbf{Outputs:} Confirmed orders, invoices, and 
        performance reports
        \item \textbf{Customers:} Walk-in customers, online buyers, 
        and regional managers
    \end{itemize}
\end{itemize}

\section*{Frame of Reference}
This framework acts as the foundation for extending ERP features, 
such as future POS integration, advanced analytics, or supplier 
management. It ensures that the current structure supports both 
day-to-day operations and long-term scalability.
